%Book
%Copyright (C) 2019  Patrick Diehl
%
%This program is free software: you can redistribute it and/or modify
%it under the terms of the GNU General Public License as published by
%the Free Software Foundation, either version 3 of the License, or
%(at your option) any later version.

%This program is distributed in the hope that it will be useful,
%but WITHOUT ANY WARRANTY; without even the implied warranty of
%MERCHANTABILITY or FITNESS FOR A PARTICULAR PURPOSE.  See the
%GNU General Public License for more details.

%You should have received a copy of the GNU General Public License
%along with this program.  If not, see <http://www.gnu.org/licenses/>.

% The template of the book is based on the The Legrand Orange Book
% LaTeX Template
% Version 2.4 (26/09/2018)
% Original author:
% Mathias Legrand (legrand.mathias@gmail.com) with modifications by:
% Vel (vel@latextemplates.com)
% License:
% CC BY-NC-SA 3.0 (http://creativecommons.org/licenses/by-nc-sa/3.0/)

%----------------------------------------------------------------------------------------
%	PACKAGES AND OTHER DOCUMENT CONFIGURATIONS
%----------------------------------------------------------------------------------------

\documentclass[11pt,fleqn]{book} % Default font size and left-justified equations

\input{structure.tex} % Insert the commands.tex file which contains the majority of the structure behind the template

%\hypersetup{pdftitle={Title},pdfauthor={Author}} % Uncomment and fill out to include PDF metadata for the author and title of the book

%----------------------------------------------------------------------------------------

\begin{document}

%----------------------------------------------------------------------------------------
%	TITLE PAGE
%----------------------------------------------------------------------------------------

\begingroup
\thispagestyle{empty} % Suppress headers and footers on the title page
\begin{tikzpicture}[remember picture,overlay]
%\node[inner sep=0pt] (background) at (current page.center) {\includegraphics[width=\paperwidth]{background.pdf}};
\draw (current page.center) node [fill=azure!30!white,fill opacity=0.6,text opacity=1,inner sep=1cm]{\Huge\centering\bfseries\sffamily\parbox[c][][t]{\paperwidth}{\centering M4997: Parallel Computaitonal Mathematics\\[15pt] % Book title
{\Large Fall 2020}\\[20pt] % Subtitle
{\huge Dr. Patrick Diehl}}}; % Author name
\end{tikzpicture}
\vfill
\endgroup

%----------------------------------------------------------------------------------------
%	COPYRIGHT PAGE
%----------------------------------------------------------------------------------------

\newpage
~\vfill
\thispagestyle{empty}

\noindent Copyright \copyright\ 2020 Patrick Diehl\orcid{0000-0003-3922-8419}\\ % Copyright notice

%\noindent \textsc{Published by Publisher}\\ % Publisher

\noindent \url{https://www.cct.lsu.edu/~pdiehl/teaching/2020/4997/}\\ % URL

\noindent \doclicenseThis  

\noindent \textit{Edition, Fall 2020} % Printing/edition date

%----------------------------------------------------------------------------------------
%	TABLE OF CONTENTS
%----------------------------------------------------------------------------------------

%\usechapterimagefalse % If you don't want to include a chapter image, use this to toggle images off - it can be enabled later with \usechapterimagetrue

\chapterimage{chapter_head_1.pdf} % Table of contents heading image

\pagestyle{empty} % Disable headers and footers for the following pages

\tableofcontents % Print the table of contents itself

\cleardoublepage % Forces the first chapter to start on an odd page so it's on the right side of the book

\pagestyle{fancy} % Enable headers and footers again

%----------------------------------------------------------------------------------------
%	PART
%----------------------------------------------------------------------------------------

\part{Introduction: C++ and the C++ standard template library}

%----------------------------------------------------------------------------------------
%	CHAPTER 1
%----------------------------------------------------------------------------------------

\chapterimage{chapter_head_2.pdf} % Chapter heading image

%----------------------------------------------------------------------------------------
\chapter{Introduction C++}
%----------------------------------------------------------------------------------------

%----------------------------------------------------------------------------------------
\section{History of C and C++}\index{Paragraphs of Text}
%----------------------------------------------------------------------------------------

%----------------------------------------------------------------------------------------
\section{Getting started with C++}
%----------------------------------------------------------------------------------------
To begin with C++ programming, we look at a simple C++ program, the so-called ``Hello World'' example which most programming language start with. Listing~\ref{code:hello:world} shows this program and the first line in green a comment is shown. A  single-line comments always starts with \lstinline[language=C++]|//| and is used to explain the functionality of the program or the next lines of codes. It is also possible to use multi-line comments\endnote{\url{https://en.cppreference.com/w/cpp/comment}} by using \lstinline[language=C++]{/* */}. Comments are important to understand the program, especially if the code is shared with other collaborators. Fore more details we refer to~\cite{kernighan1974elements}.

The second line starts with a so-called include directive\endnote{\url{https://en.cppreference.com/w/cpp/preprocessor/include}} \lstinline[language=C++]{#include <iostream>}. This include directive is needed to include functionality of the C++ standard library, see Chpater~\ref{chapter:stl}, in our case we include the \lstinline|iostream| header to print the Hello World to the terminal, see Line 6.

The fourth line \lstinline[language=C++]{int main()} starts with the so-called Main function\endnote{\url{https://en.cppreference.com/w/cpp/language/main_function}} which is the entry point of the program and all code lines within are executed sequentially one by one. Every C++ which will be compiled to an executable needs exact one function called \lstinline[language=C++]|main| which have a integer \lstinline[language=C++]{int} as its return type. On most operation systems a return value of zero means that the program executed successfully and any other value indicates an failure. The second last line \lstinline[]|return| is the so--called return statement\endnote{\url{https://en.cppreference.com/w/cpp/language/return}} which has to match the return type in front of the \lstinline[language=C++]{int main()}.


\lstinputlisting[language=C++,caption={A simple C++ program, the so-called ``Hello World'' example, which most languages start with.\label{code:hello:world}},float,floatplacement=h]{ParallelComputationMathExamples/chapter2/lecture1-1.cpp}

Once we have written the program, we have to compile the C++ code into an executable to run the code and print ``Hello world'' to the terminal. Note that there are plenty of C++ compilers\endnote{\url{https://en.wikipedia.org/wiki/List_of_compilers\#C++_compilers}} available, however this book will use the GNU Compiler Collection (GCC) for all examples. Line 1 in Listing~\ref{code:hello:world:compile} shows how to compile the file \lstinline[language=bash]|lecture1-1.cpp|, which contains the C++ code in Listing~\ref{code:hello:world}, to an executable. The GCC provides the \lstinline[language=bash]{g++} compiler to compile C++ code and the \lstinline[language=bash]{gcc} compiler to compile C code. As the first argument to the \lstinline[language=bash]{g++} compiler the file name of the C++ is provided and with the \lstinline[language=bash]|-o| option the name of the executable is specified. To run the generated executable, we type \lstinline[language=bash]|./lecture-1-1| in the terminal. Note for the basic usage of the Linux terminal we refer to~\cite{newham2005learning,robbins2016bash}.

\lstinputlisting[language=bash,caption={Compilation and execution of the C++ program.\label{code:hello:world:compile}},float,floatplacement=h,firstline=2, lastline=3]{ParallelComputationMathExamples/chapter2/lecture1-1.sh}

%----------------------------------------------------------------------------------------
\subsection{Fundamental data types}
%----------------------------------------------------------------------------------------
In this section the fundamental data types\endnote{\url{https://en.cppreference.com/w/cpp/language/types}} provided by the C++ language are introduced. First, the numeric data types are introduced. To represent natural numbers $\mathbb{N}=\{0,1,2,\ldots \}$ the \lstinline[language=C++]|unsigned int| data type is available. To represent integer numbers $\mathbb{Z}=\{\ldots,-2,-1,0,1,2,\ldots \}$ the \lstinline[language=C++]|int| data type is available. For all these data types following options: \lstinline[language=C++]|short|, \lstinline[language=C++]|long|, and \lstinline[language=C++]|long long| are available. For more details about the binary numeral system we refer to~\cite{gilli1965binary}. To represent real numbers $\mathbb{R}$ the \lstinline[language=C++]|float| data type and \lstinline[language=C++]|double| data type are available. Fore more details about the IEEE 474 standard how floating point numbers are represented in the computer we refer to~\cite{4610935,goldberg1991every}. Table~\ref{chapter2:table:datatypes} summarizes all the available numeric data types. 

\begin{table}[h]
\centering
\begin{tabular}{lccc}
\toprule
Data type & Size (Bytes) & Min & Max \\\midrule
\multicolumn{4}{c}{Natural numbers $\mathbb{N}$ }\\\midrule
\lstinline[language=C++]|unsigned short int| & 2 & 0 & 65,535  \\ 
\lstinline[language=C++]|unsigned int| & 4 & 0 & 4,294,967,295 \\ 
\lstinline[language=C++]|unsigned long int| & 4 & 0 & 4,294,967,295 \\ 
\lstinline[language=C++]|unsigned long long int| & 8 & 0 & 8,446,744,073,709,551,615 \\ \midrule
\multicolumn{4}{c}{Integer numbers $\mathbb{Z}$ }\\\midrule
\lstinline[language=C++]|short int| & 2 & -32,768 & 32,768 \\
\lstinline[language=C++]|int| & 4 & -2,147,483,648 & 2,147,483,648 \\
\lstinline[language=C++]|long long int| & 8 & $-2^{63}$ & $2^{63}-1$ \\\midrule
\multicolumn{4}{c}{Teal numbers $\mathbb{R}$ }\\\midrule
\lstinline[language=C++]|float| & 4 &  &  \\
\lstinline[language=C++]|double| & 8 &  &  \\
\bottomrule
\end{tabular} 
\caption{Overview of the fundamental numeric data types.}
\label{chapter2:table:datatypes}
\end{table}

To represent a boolean value $\mathbf{B}=\{0,1\}$ the \lstinline[language=C++]|bool| data type which has either one of the two values \lstinline[language=C++]|true| or \lstinline[language=C++]|false|.

%----------------------------------------------------------------------------------------
\subsection{Statements and flow control}
%----------------------------------------------------------------------------------------

%----------------------------------------------------------------------------------------
\subsubsection{Iteration statements}
%----------------------------------------------------------------------------------------

%----------------------------------------------------------------------------------------
\subsubsection{Selection statements}
%----------------------------------------------------------------------------------------

%----------------------------------------------------------------------------------------
\subsection{Functions}
%----------------------------------------------------------------------------------------

%----------------------------------------------------------------------------------------
\subsection{Structuring source code}
%----------------------------------------------------------------------------------------

%----------------------------------------------------------------------------------------
\subsubsection{Structs}
%----------------------------------------------------------------------------------------

%----------------------------------------------------------------------------------------
\subsection{Classes}
%----------------------------------------------------------------------------------------

\theendnotes

\chapter{The C++ Standard Template Library (STL)}
\label{chapter:stl}

\section{Overview of the STL}

\section{Working with strings}

\section{Random numbers}


\part{Linear algebra and solvers}

\chapter{Linear algebra}

\section{Blaze library}

\chapter{Solvers}


\part{Numerical examples}

\chapter{Numerical examples}

\section{Monte-Carlo methods}

\section{$N$-body problems}

\section{One-dimensional heat equation}

\section{Peridynamics}

\part{Parallel and distributes computing}

\chapter{Parallel computing}

\section{Shared memory}

\section{Distributed memory}

\part{Introduction: HPX}

\chapter{HPX}

\section{Introduction to HPX}

\section{Parallel algorithms}

\section{Asynchronous programming}





%----------------------------------------------------------------------------------------
%	BIBLIOGRAPHY
%----------------------------------------------------------------------------------------

\chapter*{Bibliography}
\addcontentsline{toc}{chapter}{\textcolor{azure}{Bibliography}} % Add a Bibliography heading to the table of contents

%------------------------------------------------

\section*{Articles}
\addcontentsline{toc}{section}{Articles}
\printbibliography[heading=bibempty,type=article]

%------------------------------------------------

\section*{Books}
\addcontentsline{toc}{section}{Books}
\printbibliography[heading=bibempty,type=book]

%----------------------------------------------------------------------------------------
%	INDEX
%----------------------------------------------------------------------------------------

\cleardoublepage % Make sure the index starts on an odd (right side) page
\phantomsection
\setlength{\columnsep}{0.75cm} % Space between the 2 columns of the index
\addcontentsline{toc}{chapter}{\textcolor{azure}{Index}} % Add an Index heading to the table of contents
\printindex % Output the index

%----------------------------------------------------------------------------------------

\end{document}
